% Created 2020-09-11 Fri 04:55
% Intended LaTeX compiler: pdflatex
\documentclass[11pt]{article}
\usepackage[utf8]{inputenc}
\usepackage[T1]{fontenc}
\usepackage{graphicx}
\usepackage{grffile}
\usepackage{longtable}
\usepackage{wrapfig}
\usepackage{rotating}
\usepackage[normalem]{ulem}
\usepackage{amsmath}
\usepackage{textcomp}
\usepackage{amssymb}
\usepackage{capt-of}
\usepackage{hyperref}
\author{rodri}
\date{\today}
\title{}
\hypersetup{
 pdfauthor={rodri},
 pdftitle={},
 pdfkeywords={},
 pdfsubject={},
 pdfcreator={Emacs 28.0.50 (Org mode 9.3.7)}, 
 pdflang={English}}
\begin{document}

\tableofcontents

\chapter{Marco Práctico}\\

\section{Esquema general del proyecto}
\label{sec:orga31ba7f}

El objetivo principal de este proyecto es la medición y grabación de señales EEG digitalmente. En esta sección del documento se detallará el proceso llevado acabo para el desarrollo de un prototipo funcional capaz de satisfacer todos los objetivos propuestos. Para la realización de un prototipo funcional se debe considerar las recomendaciones de la IFCN; alcanzar una resolución de la señal hasta 0.5µV con un rechazo de nodo común de 110dB y el ruido adicional debe ser menor de 1.5 µVpp y 0.5 µV raíz media cuadrada a cualquier frecuencia.\\

El camino más indicado, como se analizará más profundamente en los siguientes incisos, para aproximarse a estas recomendaciones es la implementación de un diseño en placa de circuito impreso con componentes de montaje superficial. Con fines de asegurar el correcto funcionamiento se realizará la implementación a través de un servicio de prototipado en planta industrial. Para ello, sin embargo, es imprescindible la simulación completa en la etapa de diseño.\\

Diseño-simulacion-prototipado-validacion DIAGRAMA\\

\begin{enumerate}
\item Que antecedentes de diseño se va a utilizar\\
\begin{enumerate}
\item Analog Front End [32ch, IFCN]\\
\item OpenBCI ADC for EEG\\
\item Power Consumption and Security [ModularEEG, Proj wireless]\\
\end{enumerate}

\item Como optimizar el diseño para modularidad\\
\begin{enumerate}
\item Entrada Electrodos OpenBCI Compatible\\
\item AFE enhancement, OpenBCI Compatible\\
\item ADC - SPI - µC-/µP\\
\item Alimentación\\
\end{enumerate}
\end{enumerate}



\section{Diseño}
\label{sec:orgee3bbe9}
\subsection{Requerimientos}
\label{sec:org586a432}
\begin{enumerate}
\item Revisión de Documentos (IFCN)\\
\end{enumerate}
\begin{center}
\begin{tabular}{lll}
Value & Consideraciones & Descripción\\
\hline
>=24 & Preferible 32 & Número de Canales\\
>=200Hz & Múltiplo de 50 o 64 & Frecuencia de Muestreo\\
>=70 Hz 12db/oct & Sample Rate (200 min) & Anti-Aliasing LP Filtro\\
\textasciitilde{}0.16 Hz & Higher Discouraged & DC-Offset HP Filtro\\
50-50 Hz & Available, not permanent & Notch Filtro\\
>=12 bit & 0.5µV Resolucion de señal & Resolucion ADC\\
>=110dB & Medido en la entrada & CMRR\\
1.5µVpp & Incluyendo 50-60 Hz & Ruido a cualquier frecuencia\\
\end{tabular}
\end{center}
\subsection{Cálculos / dimensionamiento}
\label{sec:orgb962e6e}
\begin{enumerate}
\item Selección de Topología y Arquitectura\\
\end{enumerate}
\begin{center}
\begin{tabular}{llll}
Value &  & Topologia & Descripción\\
\hline
70 Hz & 12dB/oct & MFB Butterworth & Anti-Aliasing LP Filtro\\
0.16 Hz & 8dB/oct & Sallen-Key Butterworth & DC-Offset HP Filtro\\
50-50 Hz &  & Digital & Filtro Rechaza Banda\\
2.54 &  & Ganancia en INA & Ganancia Total\\
\end{tabular}
\end{center}

\begin{enumerate}
\item Teorico\\
\end{enumerate}
\begin{center}
\begin{tabular}{lll}
Value & Topologia & Descripción\\
\hline
110dB & MFB Butterworth & CMRR\\
2.54 & Ganancia & DC-Offset HP Filtro\\
10µV & PGA x10 ADC 24bit & Resoulción LSB\\
0.16-80 &  & Banda a -20dB\\
\end{tabular}
\end{center}

\subsection{Desarrollo}
\label{sec:orgea9450c}
\subsubsection{AFE}
\label{sec:orgcbf996f}
\begin{enumerate}
\item que es afe\\
\item Diseño automatizado Analog wizard +anexos\\
\begin{enumerate}
\item Características LP 4stage\\
\item Características HP 4stage\\
\end{enumerate}
\end{enumerate}
\subsubsection{ADC}
\label{sec:org3bdf615}
\begin{enumerate}
\item Que es ADC\\
\item Open BCI módulos\\
\item Ganglion Block diagram (circuito anexo)\\
\item ADS1299 Series\\
\item ADS129x Series\\
\item *Compatibilidad\\
\item Comparativa Circuito Open BCI vs TI ADS1299\\
\item Circuito Eagle (anexo)\\
\end{enumerate}

\subsubsection{Diseño módulo de alimentación y Simulación térmica}
\label{sec:org45edac0}
\begin{enumerate}
\item Copiar diseño PWR (Modular EEG, OpenBCI, Datasheet) y simulación termica\\
\end{enumerate}
\begin{enumerate}
\item Requerimientos
\label{sec:orgaa212fe}
\begin{itemize}
\item Bateria y USB\\
\item Voltage Rails\\
\item Analyze Virtual GND\\
\item Analisis consumo\\
\end{itemize}
\begin{center}
\begin{tabular}{llrl}
Componente & Consumo & Cant & Moudlo\\
\hline
 &  & 1 & pwr\\
 &  &  & \\
 &  &  & \\
\end{tabular}
\end{center}
\item Cálculos / dimensionamiento
\label{sec:org7f2147a}
\begin{enumerate}
\item Cálculo de requerimientos; expansibilidad y flexiblidad (bateria, linea)\\
\item Grafico extrapolación potencia vs. voltaje\\
\item corriente reguladores\\
\end{enumerate}
\item Desarrollo
\label{sec:orga5c539d}
\begin{enumerate}
\item Datasheet y Open Source\\
\item Diseño Modular EEG Bloques (circuito Anexo)\\
\item diseño modular Open BCI Bloques (circuito Anexo)\\
\item Circuito unificado (circuito anexo)\\
\item MEJ02\\
\end{enumerate}
\end{enumerate}

\section{Simulación}
\label{sec:org6d1d943}
\subsection{LT Spice}
\label{sec:orge5de4b7}
\begin{enumerate}
\item Simulación LT Spice\\
\begin{enumerate}
\item Filtros PB-PA\\
\item INA\\
\item Bode\\
\item Señales Continuas (sq/sin/triang/eeg)\\
\end{enumerate}
\item Circuito Eagle (anexo)\\
\item Open BCI circuito ESD+Filtro\\
\end{enumerate}
\subsection{Eagle}
\label{sec:org5634379}
\begin{enumerate}
\item Modelado CAD\\
\begin{enumerate}
\item Diseño esquemático\\
\item Consolidación de Librería de componentes\\
\begin{enumerate}
\item Shcematic pinout\\
\item Board Packaging\\
\item Board Legends\\
\item 3D model\\
\end{enumerate}
\item Diseño Board Modular\\
\item Leyendas\\
\end{enumerate}
\end{enumerate}
\subsection{Fusion 360 - Thermal sim}
\label{sec:orgfa8c9c9}
\begin{enumerate}
\item Exportar Fusion 360\\
\item Simplificar diseño\\
\item Modelado materiales\\
\item Carga potencia\\
\item Resultados\\
\end{enumerate}
\section{Prototipado}
\label{sec:org52a8ca8}
\begin{enumerate}
\item Generación de BOM (anexo)\\
\item Cotización (anexo)\\
\item *resoulción de inconsistencias (anexo)\\
\item Panelización/Gerbers (anexo)\\
\begin{enumerate}
\item Vrails\\
\item Dimensions\\
\end{enumerate}
\item Logística y Ensamblaje\\
\item Cableado\\
\end{enumerate}
\section{Software}
\label{sec:org5fafa9b}
\subsection{Servidores}
\label{sec:org2519d8a}
\begin{itemize}
\item Custom Go Compilation\\
\item DockerFile\\
\end{itemize}
\subsection{Firmware}
\label{sec:org2241cf8}
\begin{itemize}
\item Time Tag\\
\item Dual Core (async)\\
\end{itemize}
\section{Func Gen}
\label{sec:org0d839e1}
\section{Pruebas}
\label{sec:orgf9f9a02}
\subsection{Requerimientos}
\label{sec:orgc99d3b2}
\begin{itemize}
\item Objetivos Específicos\\
\end{itemize}
\subsection{Cálculos / dimensionamiento}
\label{sec:org3094a9b}
\subsection{Desarrollo}
\label{sec:org2faf80c}
\begin{enumerate}
\item Pruebas\\
\begin{enumerate}
\item Lineas de Voltaje\\
\begin{enumerate}
\item precisión\\
\item estabilidad\\
\end{enumerate}
\item Filtros características\\
\begin{enumerate}
\item Bode\\
\item FFT\\
\item Señal Sintética Base.\\
\end{enumerate}
\item ADC\\
\begin{enumerate}
\item Comunicación SPI\\
\item Precisión voltaje de Referencia y PGA\\
\item 200Hz+ Sample rate test comparison with oscilloscope\\
\end{enumerate}
\item Software\\
\begin{enumerate}
\item MQTT Python Client Stress Test\\
\item Docker stability?\\
\item Data loss length thest\\
\end{enumerate}
\end{enumerate}
\end{enumerate}

\section{\textbf{\textbf{\textbf{Enclosure?}}}}
\label{sec:orgd12fc24}
\section{Herramientas}
\label{sec:org172074f}
\subsection{Hardware}
\label{sec:orgd5b6d76}
200 Palabras máximo por cada herramienta.\\
\subsubsection{ESP 32}
\label{sec:org937c395}
\subsubsection{USB Test Meter}
\label{sec:org411df64}
\subsubsection{DScope}
\label{sec:org4346a27}
\subsubsection{Analizador de lógica}
\label{sec:org82ad2e8}
\subsubsection{Multimetro, Cautin, Cables Dupont}
\label{sec:org99039a3}
\subsection{Software}
\label{sec:orgcdcc6b6}
200 Palabras maximo por cada herramienta.\\
\subsubsection{Analog Filter Wizard}
\label{sec:org4d889c4}
\subsubsection{Eagle CAD}
\label{sec:orgd437424}
\subsubsection{LT SPICE}
\label{sec:org87f9600}
\subsubsection{Fusion 360}
\label{sec:org3741d73}
\subsubsection{SigRok/PulseView}
\label{sec:org4f71942}
\subsubsection{Docker}
\label{sec:org1f97ce8}
\subsubsection{Go}
\label{sec:orgad1657b}
\subsubsection{Mosquitto}
\label{sec:org8d6ad29}
\subsubsection{MQTT On-Off IOS App}
\label{sec:org90373d3}
\subsubsection{EMACS arduino-cli (esptool)}
\label{sec:org3b77dcc}
\section{Resultados y discusión}
\label{sec:org3f77a4d}

\section{Análisis de Costos}
\label{sec:orgb34de3a}
\end{document}